%!TEX TS-program = xelatex
%!TEX encoding = UTF-8 Unicode
% Awesome CV LaTeX Template for Cover Letter
%
% This template has been downloaded from:
% https://github.com/posquit0/Awesome-CV
%
% Authors:
% Claud D. Park <posquit0.bj@gmail.com>
% Lars Richter <mail@ayeks.de>
%
% Template license:
% CC BY-SA 4.0 (https://creativecommons.org/licenses/by-sa/4.0/)
%


%-------------------------------------------------------------------------------
% CONFIGURATIONS
%-------------------------------------------------------------------------------
% A4 paper size by default, use 'letterpaper' for US letter
\documentclass[11pt, a4paper]{awesome-cv}

% Configure page margins with geometry
\geometry{left=1.4cm, top=.8cm, right=1.4cm, bottom=1.8cm, footskip=.5cm}

% Specify the location of the included fonts
\fontdir[fonts/]

% Color for highlights
% Awesome Colors: awesome-emerald, awesome-skyblue, awesome-red, awesome-pink, awesome-orange
%                 awesome-nephritis, awesome-concrete, awesome-darknight
\colorlet{awesome}{awesome-skyblue}
% Uncomment if you would like to specify your own color
% \definecolor{awesome}{HTML}{CA63A8}

% Colors for text
% Uncomment if you would like to specify your own color
% \definecolor{darktext}{HTML}{414141}
% \definecolor{text}{HTML}{333333}
% \definecolor{graytext}{HTML}{5D5D5D}
% \definecolor{lighttext}{HTML}{999999}

% Set false if you don't want to highlight section with awesome color
\setbool{acvSectionColorHighlight}{true}

% If you would like to change the social information separator from a pipe (|) to something else
\renewcommand{\acvHeaderSocialSep}{\quad\textbar\quad}
\usepackage[ngerman, english]{babel}

%-------------------------------------------------------------------------------
%	PERSONAL INFORMATION
%	Comment any of the lines below if they are not required
%-------------------------------------------------------------------------------
% Available options: circle|rectangle,edge/noedge,left/right
%\photo[circle,noedge,left]{profile.jpg}
\name{Simon}{Symhoven}
\position{Student der Wirtschaftsinformatik{\enskip\cdotp\enskip}Werkstudent{\enskip\cdotp\enskip}Gründer}
\address{Boschetsriederstraße 59a, 81379 München, Deutschland}

\mobile{(+49) 176 67840763}
\email{post@simon-symhoven.de}
%\homepage{simon-symhoven.de}
\github{simonsymhoven}
\linkedin{simonsymhoven}
% \gitlab{gitlab-id}
% \stackoverflow{SO-id}{SO-name}
% \twitter{@twit}
% \skype{skype-id}
% \reddit{reddit-id}
% \medium{madium-id}
% \googlescholar{googlescholar-id}{name-to-display}
%% \firstname and \lastname will be used
% \googlescholar{googlescholar-id}{}
% \extrainfo{extra informations}

%\quote{``Be the change that you want to see in the world."}


%-------------------------------------------------------------------------------
%	LETTER INFORMATION
%	All of the below lines must be filled out
%-------------------------------------------------------------------------------
% The company being applied to
\recipient
  {Ludwig-Maximilians-Universität München}
  {Geschwister-Scholl-Platz 1\\80539 München}
% The date on the letter, default is the date of compilation
\letterdate{{\selectlanguage{ngerman}\today}}
% The title of the letter
\lettertitle{Motivationsschreiben Master-Studium Informatik}
% How the letter is opened
\letteropening{Sehr geehrte Damen und Herren,}
% How the letter is closed
\letterclosing{Mit freundlichen Grüßen}
% Any enclosures with the letter
\letterenclosure[Anhang]{Curriculum Vitae}


%-------------------------------------------------------------------------------
\begin{document}

% Print the header with above personal informations
% Give optional argument to change alignment(C: center, L: left, R: right)
\makecvheader[R]

% Print the footer with 3 arguments(<left>, <center>, <right>)
% Leave any of these blank if they are not needed
\makecvfooter
  {\selectlanguage{ngerman}\today}
  {Simon Symhoven~~~·~~~Motivationsschreiben}
  {}

% Print the title with above letter informations
\makelettertitle

%-------------------------------------------------------------------------------
%	LETTER CONTENT
%-------------------------------------------------------------------------------
\begin{cvletter}

ich stelle mich als Kandidat für den Master-Studiengang in \glqq{}Informatik\grqq{} an Ihrer Universität vor.

Im Oktober 2017 begann ich das Bachelor-Studium \glqq{}Informatik + Mathematik\grqq{} an der Ludwig-Maximilians Universität. Nach dem zweiten Semester entschloss ich mich, den Studiengang zu wechseln und startete daraufhin im Oktober 2018 das Bachelor-Studium \glqq{}Wirtschaftsinformatik\grqq{} an der Hochschule München, welches ich voraussichtlich im August mit dem B.Sc. gut abschließen werde. 

Der Wechsel lag vor allem darin begründet, dass ich mir von dem Studium an der Hochschule einen größeren Praxisbezug erhoffte. Dies wurde erfüllt, allerdings fehlt mir nun nach sechs Semestern der theoretische Tiefgang in meinem Studium. Das Bachelor-Studium ist sicherlich eine wesentliche Säule, durch meine gesammelte Erfahrung der letzten vier Jahre als Werkstudent, Freelancer und Gründer einer eigenen Unternehmergesellschaft kamen der Spaß und die Begeisterung an komplexeren Zusammenhängen und Themen außerhalb der an der Hochschule und Universität gelehrten Fachgebiete. 

Ich bin mir sicher, mit dieser Studienwahl die richtige Entscheidung getroffen zu haben. In der Bewältigung des Studiums sehe ich keine Probleme. Die Mathematik ist mir bisher immer leicht gefallen; die Grundlagen habe ich an der LMU absolviert, weshalb ich mir deshalb den hohen Anforderungen und der großen Erwartungshaltung bewusst bin. Hohe Anforderungen sind gut und ich verstehe das als Herausforderung.  

Durch die Berufserfahrung und privaten Projekte, einschließlich der damit zusammenhängenden Erfolge der vergangenen Jahre, habe ich viele Kenntnisse und Erfahrungen in diversen Programmiersprachen, DevOps, Tools, Technologien und vor allem den sozialen Kompetenzen erlangt. Ich möchte meine Interesse und Neugier weitertreiben, mich spezialisieren und die Dinge in der Tiefe behandeln und notwendigerweise diskutieren.

Ich freue mich von Ihnen zu hören!
\end{cvletter}


%-------------------------------------------------------------------------------
% Print the signature and enclosures with above letter informations
\makeletterclosing

\end{document}
